
\documentclass{article}

\usepackage{graphicx}
\usepackage{hyperref}
\usepackage{tikz}
\usepackage[utf8]{inputenc}
\usepackage[T1]{fontenc}
\usepackage{listings}

\title{Projet programmation 2\\Phase 3}
\author{R\'emi Oudin, Alexis Laouar, K\'evin Le Run}
\date{}

\begin{document}

\maketitle
\section{Améliorations de l'interface graphique}

L'organisation des menus a été totalement réécrite. Ceci avait pour but principal
de pouvoir créer des menus et de naviguer entre ceux-cis. Pour ce faire
nous avons utilisé une \emph{state machine}. Chaque menu représente un état de
la \emph{state machine}, et le jeu n'est plus qu'un état de la \emph{state machine}.

De plus, l'organisation dans chaque menu se fait de manière arborescente. Chaque
bouton, champ,\dots est un fils de la \emph{gui}.

\lstinputlisting{menu.scala}

\section{Mise en place d'un réseau}

\subsection{Le réseau}
Désormais, le jeu se déroule en réseau. Celui-ci est un réseau de type Client-Serveur,
c'est-à-dire que certaines communications sont gérées entre les clients et le
serveur qui sert de relais pour transmettre des informations telles que "Player $i$
a posé une tour $t$ en position $(x,y)$. De plus, le serveur effectue une synchronisation
toutes les 5 secondes, pendant laquelle le serveur envoie toutes les informations
nécessaires au jeu à tous les clients. Pendant cette étape, il est maître, c'est-à-dire
que les clients ne remettront pas en cause les données qu'il envoie. Pendant le reste du temps,
le serveur va laisser les clients prédire l'évolution du jeu.

\subsection{Les nouvelles mécaniques de jeu}

Nous avons créé un système de stratégies pour que le client et le serveur aient quasiment
le même code. Celle-ci sont des fonctions de l'état \emph{GameState} et permettent
de modifier les quelques morceaux de code qui diffèrent, c'est-à-dire lorsque les
joueurs ont des informations à faire transiter.\\
De plus, le jeu se déroule de manière symétrique. Chaque joueur va devoir défendre
sa zone, et attaquer les autres joueurs grâce à des ennemis et des spawners qu'il va
pouvoir acheter.

\section{Les spawners}

Les spawners sont des tours qui ont comme mécanique d'attaque le spawn d'ennemis.
Ils utilisent un équivalent de structure de liste chaînée pour pouvoir boucler à
l'infini.

\end{document}
